\documentclass{article}
\usepackage[centertags]{amsmath}
\usepackage{amsfonts}
\usepackage{amssymb}
\usepackage{amsthm}
\usepackage{amsmath}
\usepackage[textwidth=15cm]{geometry}

\usepackage{tabularx} % Paket za tabele
\usepackage{subfigure} % Paket za vi�e figura u jednoj
\usepackage{tikz} % Paket za crtanje

\newtheorem{zadatak}{Zadatak}
\newtheorem{primjer}{Primjer}

\newenvironment{rjesenje}
{\textbf{Rješenje:}} {\hfill $\clubsuit$}

\begin{document}
% U preambuli su navedena pored standardnih paketa i tri paketa neophodna za naredna pisanja.

% PRIMJER RADA Zadatka
\begin{zadatak}
    Dokazati da
    $(A\setminus B)\cup (A\cap B)\cup (B\setminus A)=A\cup B$\\
\end{zadatak}

\vskip 1cm

% PRIMJER RADA Primjera
\begin{primjer}
Neka su $A,B,C$ i $D$ proizvoljni skupovi. U kom odnosu su skupovi $(A\cap B)\cup (C\cap D)$ i $(A\cup C)\cap (B\cup D)$?
\end{primjer}
\begin{rjesenje}
Za proizvoljno $x$ vrijedi:
\begin{align*}
x \in (A\cap B)\cup (C\cap D) & \Longleftrightarrow x\in A\cap B \lor x\in C\cap D \\
& \Longleftrightarrow (x\in A \land x\in B) \lor (x\in C \land x\in D) \\
& \Longrightarrow (x\in A \lor x\in C) \land (x\in B \lor x\in D) \\
& \Longleftrightarrow x\in (A\cup C)\cap (B\cup D) \ .
\end{align*}
Dakle, vrijedi $(A\cap B)\cup (C\cap D) \subseteq (A\cup C)\cap (B\cup D)$.\\
Da li vrijedi jednakost u ovoj vezi? Odgovor je NE! Primjer za to su skupovi $A=\{0\}$, $B=\{1\}$, $C=\{1\}$ i $D=\{0\}$. Sada imamo
$$A\cap B = \varnothing \ , \ C\cap D=\varnothing \ , \ A\cup C =\{0,1\} \ , \ B\cup D=\{0,1\} \ .$$
Dakle,
$$(A\cap B)\cup (C\cap D)=\varnothing \ , \ (A\cup C)\cap (B\cup D)=\{0,1\} \ ,$$
to jest
$$(A\cap B)\cup (C\cap D)=\varnothing \subset \{0,1\} = (A\cup C)\cap (B\cup D) \ .$$
\end{rjesenje}

\vskip 1cm

% PRIMJER RADA SA Tabelama
\begin{tabularx}{13cm}{|c|c|c|c|c|c|c|c|c|c}
$A$ & $B$ & $C$ & $A\setminus C$ & $B\setminus C$ & Lijevo & $A\cap B$ & Desno & Venn & \\\hline
Ne & Ne & Ne & Ne & Ne & Ne & Ne & Ne & 0 & \\\hline
Da & Ne & Na & Da & Ne & Ne & Ne & Ne & 4 & \\\hline
Ne & Da & Ne & Ne & Da & Ne & Ne & Ne & 2 & \\\hline
Ne & Ne & Da & Ne & Ne & Ne & Ne & Ne & 1 & \\\hline
Da & Da & Ne & Da & Da & Da & Da & Da & 6 & \checkmark \\\hline
Da & Ne & Da & Ne & Ne & Ne & Ne & Ne & 5 & \\\hline
Ne & Da & Da & Ne & Ne & Ne & Ne & Ne & 3 & \\\hline
Da & Da & Da & Ne & Ne & Ne & Da & Ne & 7 & \\\hline
\end{tabularx}

\vskip 1cm

% PRIMJER NENUMERISANOG IZVO�ENjA DOKAZA SA KOMENTARIMA
\begin{align*}
x\in A\cup B &\Longrightarrow  x\in A \lor x\in B    \tag{\textrm{\footnotesize aksiom unije}}\\
&\Longrightarrow  (x\in A \land x\in B)\lor x\in B   \tag{\textrm{\footnotesize pretpostavka $A=A\cap B$}}\\
&\Longrightarrow x\in B                              \tag{\textrm{\footnotesize tautologija $((p\land q)\lor q)\Rightarrow q $}}
\end{align*}

\vskip 1cm

% PRIMJER ZA VENNOVE DIJAGRAME u TIKZ paketu
\begin{figure}[h]
\subfigure[]{
\begin{tikzpicture}[scale=0.6]
    \draw (-2,2) rectangle (4.4cm,-2cm);
    \draw (0,0) circle (1.5cm);
    \draw (2.25,0) circle (1.5cm);
    \node at (0,0)  (a)    {\scriptsize $2$};
    \node at (1.1,0)  (b)    {\scriptsize $3$};
    \node at (2.5,0)  (c)    {\scriptsize $1$};
    \node at (-1.6,1.7)  (d)    {\scriptsize $0$};
\end{tikzpicture}} \hskip 2cm
\subfigure[]{
\begin{tikzpicture}[scale=0.6]
    \draw (-2,2) rectangle (4.4cm,-3.7cm);
    \draw (0,0) circle (1.5cm) node {};
    \draw (2.25,0) circle (1.5cm) node {};
    \draw (1.25,-1.75) circle (1.5cm) node {};
    \node at (0,0)  (a) {\scriptsize $4$};
    \node at (2.25,0)  (b) {\scriptsize $2$};
    \node at (1.25,-1.75)  (c) {\scriptsize $1$};
    \node at (1.8,-1)  (d) {\scriptsize $3$};
    \node at (1.15,0)  (e) {\scriptsize $6$};
    \node at (0.5,-1)  (f) {\scriptsize $5$};
    \node at (1.1,-0.6)  (e) {\scriptsize $7$};
     \node at (-1,-3)  (g) {\scriptsize $0$};
\end{tikzpicture}}
\end{figure}


\end{document} 